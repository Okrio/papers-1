\documentclass[a4paper]{article}
\usepackage{INTERSPEECH2014,amssymb,amsmath,epsfig,url}
\setcounter{page}{1}
\sloppy     % better line breaks
\ninept
%SM below a registered trademark definition
\def\reg{{\rm\ooalign{\hfil
     \raise.07ex\hbox{\scriptsize R}\hfil\crcr\mathhexbox20D}}}

%% \newcommand{\reg}{\textsuperscript{\textcircled{\textsc r}}}
\title{Robust speech recognition using temporal masking and thresholding algorithm}

%%%%%%%%%%%%%%%%%%%%%%%%%%%%%%%%%%%%%%%%%%%%%%%%%%%%%%%%%%%%%%%%%%%%%%%%%%
%% If multiple authors, uncomment and edit the lines shown below.       %%
%% Note that each line must be emphasized {\em } by itself.             %%
%% (by Stephen Martucci, author of spconf.sty).                         %%
%%%%%%%%%%%%%%%%%%%%%%%%%%%%%%%%%%%%%%%%%%%%%%%%%%%%%%%%%%%%%%%%%%%%%%%%%%
%\makeatletter
%\def\name#1{\gdef\@name{#1\\}}
%\makeatother
%%%%%%%%%%%%%%% End of required multiple authors changes %%%%%%%%%%%%%%%%%

\makeatletter
\def\name#1{\gdef\@name{#1\\}}
\makeatother \name{{\em Chanwoo~Kim$^1$, Kean~K.~Chin$^1$, Michiel~Bacchiani$^1$, Richard~M.~Stern$^2$}}

\address{Google, Mountain View CA 94043 USA $^1$\\
  Carnegie Mellon University, Pittsburgh PA 15213 USA $^2$\\
{\small \tt \{chanwcom, kkchin, michiel\}@google.com, rms@cs.cmu.edu}}

%\twoauthors{Karen Sp\"{a}rck Jones.}{Department of Speech and Hearing \\
%  Brittania University, Ambridge, Voiceland \\
%  {\small \tt Karen@sh.brittania.edu} }
%  {Rose Tyler}{Department of Linguistics \\
%  University of Speechcity, Speechland \\
%  {\small \tt RTyler@ling.speech.edu} }

%
\usepackage{tikz}
\tikzstyle{every picture}+=[font=\rmfamily\it\bfseries\large]
\usepackage[caption=false,font=footnotesize]{subfig}
\begin{document}
\maketitle
%
\begin{abstract}
In this paper, we present a new dereverberation algorithm called
 Temporal Masking and Thresholding (TMT) to enhance the temporal
 spectra of spectral features for robust speech recognition
 in reverberant environments. This algorithm is motivated
 by the precedence effect and temporal masking of human
 auditory perception. This work is an improvement
 of our previous dereverberation work called Suppression of 
 Slowly-varying components and the falling edge of the power
 envelope (SSF).  The TMT algorithm  uses a different mathematical
 model to characterize temporal masking and thresholding compared
 to the model that had been used to characterize the SSF algorithm.
 Specifically, the nonlinear highpass filtering used in the SSF
 algorithm has been replaced by a masking mechanism
 based on a combination of peak detection and dynamic thresholding.
 Speech recognition results show that the TMT algorithm provides
 superior recognition accuracy compared to other algorithms such
 as LTLSS, VTS, or SSF in reverberant environments.

\end{abstract}
\noindent{\bf Index Terms}: Robust speech recognition, speech enhancement, reverberation,
temporal masking, precedence effect


%
\section{Introduction}

\begin{eqnarray}
  y[n] & = & x[n] * h[n]  \nonumber \\
  \log (Y[m, e^{j \omega_k})) & = & \log(X[m, e^{j \omega_k}))  \nonumber \\
            && + \log(H[m, e^{j \omega_k}))  \nonumber
\end{eqnarray}

\begin{eqnarray}
 \log   \left(\Big| Z[m, e^{j \omega_k}) \Big| \right)  & = &  \log \left(\Big| Y[m, e^{j \omega_k}) \Big| \right)  
  - \frac{1}{M} \log \left( \sum_{m = 0}^{M - 1} \Big| Y[m, e^{j \omega_k}) \Big| \right) \nonumber \\
&=& \log \left(\Big| X[m, e^{j \omega_k}) \Big| \right) 
 - \frac{1}{M} \log \left( \sum_{m = 0}^{M - 1} \Big| X[m, e^{j \omega_k}) \Big| \right) \nonumber
\end{eqnarray}


\section{Conclusion}
%TODO(chanwcom)
% Revise the conclusion
In this paper, we describe a new dereverberation algorithm,
TMT, that is based on temporal enhancement by estimating the
peak sound level and applying the temporal masking. In the
experimental results, we have observed that TMT algorithm
is simple, but has shown better speech recognition accuracies
than existing algorithms like LTLSS or VTS. The matlab code
for this algorithm may be found at 
\url{http://www.cs.cmu.edu/~robust/archive/algorithms/tmt}.

% use section* for acknowledgement
\section{Acknowledgements}
This research was supported by Google.

\end{document}
